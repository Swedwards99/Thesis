\documentclass[12pt]{extarticle}
\usepackage[utf8]{inputenc}
\usepackage{natbib}
\bibliographystyle{apa}

\title{Transgenerational Developmental and Behavioral Plasticity of Threespine Stickleback \textit{Gasterosteus aculeatus} of Lake Myvatn, Iceland}
\author{Spencer Edwards, Roll No.}
\date{April 2021}

\begin{document}

\maketitle

\section*{Introduction}
Much research has been conducted exploring phenotypic plasticity, the ability for an organism to alter its phenotype in response to its environment within its lifetime \citep{Denver2010, Kishida2010, Klemetsen2010}. However, there is increasing interest in so-called transgenerational plasticity, wherin plastic responses to environmental conditions are passed down to offspring \citep{Hellmann2020, Richter-Boix2014, Bell2019, Shama2014}. This concept holds particular importance in the face of rapid environmental change, because it allows organisms to evolve without relying on actual genetic changes. One of the most well-known model systems for studying plasticity is \textit{Gasterosteus aculeatus}, commonly named the threespine stickleback. Stickleback populations often form species pairs, with two or more morphs coexisting in a single body of water via niche partioning. In lake Myvatn, two morphs exist, the ``mud'' and ``lava'' morphs, defined mainly by habitat type, as well as morphology and behavior \citep{Kristjansson2002, Millet2013}. The fish face multiple  

Proposed Guides: Dr. Bjarni Kristj\'ansson, Department of Aquaculture and Fish Biology, H\'olar University College.
\bibliography{mainbib}

\end{document}

