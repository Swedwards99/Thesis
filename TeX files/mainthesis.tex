\documentclass[12pt]{extarticle}
\usepackage[utf8]{inputenc}
\usepackage[T1]{fontenc}
\usepackage[english]{babel}
\usepackage{natbib}
\usepackage{gensymb}
\bibliographystyle{apa}

\title{Parental Effects on Development and Behavior in Threespine Stickleback \textit{Gasterosteus aculeatus} of Lake M\'yvatn, Iceland}
\author{Spencer Edwards, Roll No.}
\date{April 2021}

\begin{document}

\maketitle

\section*{Introduction}
<<<<<<< HEAD
Development and early life history are two of the most important time periods for an organism. Consequentally, the traits gained by an organism from its parents during this time are key to its success as an individual. It is not suprising then, that parental effects, both maternal and paternal, have a major impact on the an organism from before it is born \citep{charmantier_garant_kruuk_2014, Danchin2011, Badyaev2009}. Parental effects, defined as a change in an Offspring's phenotype due to the genotype and/or environment of one or more of its parents, were for a long time considered a nuicance, a deviation from pure heritable traits. Now however we have seen a researgence of interest in these effects \citep{charmantier_garant_kruuk_2014}. While much research has been done concerning the impact of parental effects on development \citep{Tigreros2021} and behavior of invertabrates (particularly beetles), comparitivley less is known about parental effects on vertabrates. Work on teleosts has revealed strong links between parental effects on the co-evolution of behavior and cellular function \citep{Yoshizawa2012}. The connection between parental affects influencing behavior and gene expression has serious implications for evolutionary ecology, because rapid environmental could act as a mechanism for rapid evolution via parental programming of offspring phenotypes, effectively providing their offspring with a ``jumpstart'' of sorts \citep{Danchin2011, Donelson2018}. 
From the standpoint of quantitative genetics, the phenotype (\textit{P}) is equal to the influence of genes (\textit{G}) and environment (\textit{E}) such that $$P = G \times E $$

However, genetic effects from parents can be further broken down into inheritence from genes directly ($I_G$) and inheritence from the parental environment ($I_E$) such that $$P = (I_G \times I_E) \times E$$

Here in Iceland, work on various populations of \textit{Gasterosteus aculeatus} has focused on phenotypic diversity, including phenotypic plasticity \citep{Kristjansson2002, Millet2013}. However, the extent to which parental effects shape the morphology and behavior of lake M\'yvatn stickleback has yet to be investigated. Previous work on different stickleback populations have discovered a range of parental effects, both maternal and paternal. \citet{Bell2018} investigated the heritability of parental behavior (Specifically, fanning of eggs) by male stickleback and found strong heritability of the trait. Furthermore, they concluded that a strong amount of genetic variation could lead the evolvability of the fanning trait. Offspring of male stickleback that expirience predation risk have shown to grow smaller and spend less time in the open \citep{Bell2016, Stein2014} Female stickleback have also been shown to pass on phenotypic information to their eggs. An RNAseq study analyzing female sitckleback found that eggs from mothers exposed to predators had faster development times, as well as major epigenetic changes and alterations to non-coding genes during development \citep{Mommer2014, Bell2016}.  \\


Major questions I want to address:
\begin{itemize}
 \item To what extent are the phenotypes of lake M\'yvatn stickleback 
shaped by maternal and paternal (parental) effects?
 \item Specifically, how much variation in stickleback phenotypes are driven by genetic inheritence vs environmental inheritence?
 \item Do M\'yvatn stickleback 
=======
Much research has been conducted exploring phenotypic plasticity, the ability for an organism to alter its phenotype in response to its environment within its lifetime \citep{Denver2010, Kishida2010, Klemetsen2010}. However, there is increasing interest in so-called transgenerational plasticity (TGP), wherin plastic responses to environmental conditions are passed down to offspring \citep{Hellmann2020, Richter-Boix2014, Bell2019, Shama2014}. This concept holds particular importance in the face of rapid environmental change, because it allows organisms to evolve without relying on actual genetic changes. One of the most well-known model systems for studying plasticity is \textit{Gasterosteus aculeatus}, commonly named the threespine stickleback. Stickleback populations often form species pairs, with two or more morphs coexisting in a single body of water via niche partioning. In lake Myvatn, two morphs exist, the ``mud'' and ``lava'' morphs, defined mainly by habitat type, as well as morphology and behavior \citep{Kristjansson2002, Millet2013}. Different sections of lake Myvatn act as distinct habitats (referred to as the North, South, and (sometimes) East basins), and so the sticklebacks can also be distinguished by habitat of origin \citep{Millet2013, Einarsson2004}. \citet{Millet2013} in particular found that there was a large difference in size between sticklebacks from different habitats, with those of the North basin being larger ($\ge$ 55mm) and of different age classes than those of the South. They posit that these difference could be due to differences in life history strategies between the populations, or plasticity in response to resources. Interestingly, they also note that North basin stickleback have larger spines than those of the South basin, possibly owing to differences in abundance of gape-limited predators, which occur at rates of up ten times higher in the North basin than the South \citep{Millet2013}. While we know that plasticity in traits does occur in lake Myvatn stickleback, there is little data on whether this plasticity is transgenerational. Indeed, much of the research on TGP in stickleback has been performed on oceanic populations, and thus TGP in freshwater stickleback represents a current gap in our knowledge of the ecology and evolution of these organisms \citep{Shama2014}. Evidence from marine stickleback suggests that TGP does occur, although it varies between populations \citep{Shama2014, Heckwolf2018, Kozak2012}. Research by \citet{Kozak2012} found that TGP led to increased shoaling behavior in response to predators by limnetic morphs of stickleback in British Columbia, while benthic morphs did not show evidence of TGP. This is particularly interesting in the context of lake M\'yvatn, because as noted above, different subpopulations of stickleback occur in regions with different predator abundances. Particularly in the North basin, where predator abundance is much greater than the South, we could expect to see TGP acting to alter antipredator behavior. Other research has implications for responses to climate change. \citet{Heckwolf2018} investigated adaptive and nonadaptive TGP in marine stickleback in response to changes in salinty. They found that directional selection of nonadaptive TGP accelarated the evolutionary response in offspring, but that it was dependent on life stage and the particular environmental factor they are exposed to. In the context of lake M\'yvatn, the different basins apply different environmental factors to the sticklebacks living in them (in terms of different temperatures, prey and predator abundances, and seasonal dynamics) and thus the evolutionary selection on TGP might be altered.   

Major questions I want to address:
\begin{itemize}
 \item Do lake M\'yvatn stickleback experience morphological/behavioral TGP in response to food and predator presence, temperature, and salinty? Are maternal or paternal effects stronger?
 \item Does TGP derive from mothers, fathers, or both?
>>>>>>> 74244923d2cdea290fcd5aba989f380fa10b2c0c
\end{itemize}


\section*{Methods}
I will use a half-sibiling common garden expiriment to assess TGP in M\'yvatn stickleback. Stickleback will be collected from seperate basins of the lake, the North, or cold shore, and the South, or warm shore. Sticklebacks from both shores will then be crossed using standard methods laid out by Schluter. Half-sibling split clutch designs have been used previously to examine the effects of indirect fitness in parasite resistance, as well as transgenerational plasticity in response to temperature in marine stickleback \citep{Barber2001,Ramler2014}. One of the major problems when considering historical work on parental effects is disentangling genetic and environmental effects \citep{Donelson2018}. Using a Linear Mixed Model (LMM) approach to decompile the effects should be effective.
15 males and 15 breeding females will be collected from the shore of the hot basin (hereon: hot shore) and a nearby cool region called Grimstaðir. Sperm from each male will be split and used to fertilize two females, and eggs from each female will be fertilized by two males. Each family will be split between two tanks to account for tank effects on individuals. Fish will be
\\

Proposed Guides: Dr. Bjarni Kristj\'ansson, Department of Aquaculture and Fish Biology, H\'olar University College.
\bibliography{mainbib}

\end{document}

