\documentclass[12pt]{extarticle}
\usepackage[utf8]{inputenc}
\usepackage{natbib}
\usepackage{gensymb}
\bibliographystyle{apa}

\title{Transgenerational Developmental and Behavioral Plasticity of Threespine Stickleback \textit{Gasterosteus aculeatus} of Lake Myvatn, Iceland}
\author{Spencer Edwards, Roll No.}
\date{April 2021}

\begin{document}

\maketitle

\section*{Introduction}
Much research has been conducted exploring phenotypic plasticity, the ability for an organism to alter its phenotype in response to its environment within its lifetime \citep{Denver2010, Kishida2010, Klemetsen2010}. However, there is increasing interest in so-called transgenerational plasticity (TGP), wherin plastic responses to environmental conditions are passed down to offspring \citep{Hellmann2020, Richter-Boix2014, Bell2019, Shama2014}. This concept holds particular importance in the face of rapid environmental change, because it allows organisms to evolve without relying on actual genetic changes. One of the most well-known model systems for studying plasticity is \textit{Gasterosteus aculeatus}, commonly named the threespine stickleback. Stickleback populations often form species pairs, with two or more morphs coexisting in a single body of water via niche partioning. In lake Myvatn, two morphs exist, the ``mud'' and ``lava'' morphs, defined mainly by habitat type, as well as morphology and behavior \citep{Kristjansson2002, Millet2013}. Different sections of lake Myvatn act as distinct habitats (referred to as the North, South, and (sometimes) East basins), and so the sticklebacks can also be distinguished by habitat of origin \citep{Millet2013, Einarsson2004}. \citet{Millet2013} in particular found that there was a large difference in size between sticklebacks from different habitats, with those of the North basin being larger ($\ge$ 55mm) and of different age classes than those of the South. They posit that these difference could be due to differences in life history strategies between the populations, or plasticity in response to resources. Interestingly, they also note that North basin stickleback have larger spines than those of the South basin, possibly owing to differences in abundance of gape-limited predators, which occur at rates of up ten times higher in the North basin than the South \citep{Millet2013}. While we know that plasticity in traits does occur in lake Myvatn stickleback, there is little data on whether this plasticity is transgenerational. Indeed, much of the research on TGP in stickleback has been performed on oceanic populations, and thus TGP in freshwater stickleback represents a current gap in our knowledge of the ecology and evolution of these organisms \citep{Shama2014}. Evidence from marine stickleback suggests that TGP does occur, although it varies between populations \citep{Shama2014, Heckwolf2018, Kozak2012}. Research by \citet{Kozak2012} found that TGP led to increased shoaling behavior in response to predators by limnetic morphs of stickleback in British Columbia, while benthic morphs did not show evidence of TGP. This is particularly interesting in the context of lake M\'yvatn, because as noted above, different subpopulations of stickleback occur in regions with different predator abundances. Particularly in the North basin, where predator abundance is much greater than the South, we could expect to see TGP acting to alter antipredator behavior. Other research has implications for responses to climate change. \citet{Heckwolf2018} investigated adaptive and nonadaptive TGP in marine stickleback in response to changes in salinty. They found that directional selection of nonadaptive TGP accelarated the evolutionary response in offspring, but that it was dependent on life stage and the particular environmental factor they are exposed to. In the context of lake M\'yvatn, the different basins apply different environmental factors to the sticklebacks living in them (in terms of different temperatures, prey and predator abundances, and seasonal dynamics) and thus the evolutionary selection on TGP might be altered.   

Major questions I want to address:
\begin{itemize}
 \item Do lake M\'yvatn stickleback experience morphological/behavioral TGP in response to food and predator presence, temperature, and salinty? Are maternal or paternal effects stronger?
 \item Does TGP derive from mothers, fathers, or both?
\end{itemize}


\section{Methods}
I will use a half-sibiling common garden expiriment to assess TGP in M\'yvatn stickleback.

Proposed Guides: Dr. Bjarni Kristj\'ansson, Department of Aquaculture and Fish Biology, H\'olar University College.
\bibliography{mainbib}

\end{document}

