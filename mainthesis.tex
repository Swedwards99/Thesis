\documentclass[12pt]{extarticle}
\usepackage[utf8]{inputenc}
\usepackage{cite}

\title{Impact of Developmental Plasticity of Three Spine Stickleback \textit{Gasterosteus aculeatus} on the Community Structure of Lake M\'yvatn, Iceland}
\author{Spencer Edwards, Roll No.}
\date{April 2021}

\begin{document}

\maketitle

Evolution and ecology exert multiple forces upon indivudials and ecosystems in response to enviormental pressures, the complexities of which are only beginning to be understood. Two of these forces are phenotypic variation, derived from genes or plasticity; and ecological succession, the process by which community strucutres in an ecosystem change over time. These two forces act reciprically to produce the variation in organisms at the individual level, all the way to the level of communities and ecosystems. One of the best model systems for studying the interaction between evolution and ecology is \textit{Gasterosteus aculeatus}, commonly referred to as the Three Spine Stickleback (TSS). TSS often form species pairs, wherein niche partioning drives morphological and behavioral differences - often along the so-called ``Benthic-Limnetic axis''. In lake M\'yvatn, TSS form two morphs: The benthic, or ``mud'' morph, and the limnetic, or ``lava'' morph. The mud morph, much like other benthic morphs, is smaller and has a mouth suited for grabbing small cladocerans from the bottom.The lava morph is larger, and has a mouth suited for consuming plankton and, importantly, Chironomid midges, of which the lake gets its name. TSS are, in turn, consumed by salmonoids, namely \textit{Salvelinus alpinus}, Arctic char and \textit{Salmo trutta}, brown trout; as well as the many species of waterfowl that inhabit the lake.

The level to which TSS morphological diversity is driven by genetic or phenotypic plasticity is still debated, however TSS have shown to experience developmental plasticity in response to food availability, including in lake M\'vatn.

And in the last paragraph write about the proposed deliverable aimed from this thesis  \cite{fractalwiki}.

Proposed Guides: Dr./Mr./Ms Name of faculty, Professor/Associate/Assistant Professor, Department, Campus.  \nocite{higham1998handbook}


\bibliographystyle{plain}
\bibliography{M335}

\end{document}
